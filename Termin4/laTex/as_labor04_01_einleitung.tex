\section{Einleitung und Ziel}

Im 4. Labor im Modul Aktorik und Sensorik soll ... bestimmt werden.


Was müssen Sie um das bestehende Motormodell hinzufügen, um einen Strom einprägen zu können?
H-Brücke

2.1
T_LH und T_RL

2.2 
Durch die Drehung des Motors wird eine Spannung Induziert. Dadurch fließt ein Strom durch die 
sogenannten Parasitärdioden T_RH und T_LL. 

2.3
Der Strom lässt sich über R_shunt messen, welcher passenderweise mit 1 OHM gewählt ist. So ist
die Spannung, die an R_shunt abfällt gleich des Stromes.

2.4
Es gibt vier verschiedene Zustände. Wenn der Motor im Uhrzeigersinn betrieben werden soll, können
die entsprechenden Transistoren geschaltet sein oder nicht geschaltet sein. Soll der Motor gegen
den Uhrzeigersinn betrieben werden, kann das andere Transistorenpaar geschaltet sein oder nicht.

2.5
Eingänge: I_IST, I_Soll, U_H-Brücke
Ausgänge: U_Motor

2.6
1. Fall: AN
    U_H = U_T_LH + U_Motor + U_T_RL + U_R_shunt

    U_H = R_LH * I_m + (R_shunt * I_m)  

2. Fall: Aus
    0 = U_H + U_D + U_Motor + U_D + U_shunt