\section{Messung des Stillstandsdrehmomentes}

\subsection{Beschreibung}

Im zweiten Versuch soll der Ankwiderstand $R$ bestimmt werden.
Der Ankwiderstand kann über das Ohm'sche Gesetz berechnet werden,
dafür ist eine Matrix mit den Messwerten der Spannungen und Ströme gegeben.

\begin{equation} \label{eq2}
    \begin{split}
        R=\frac{U_a}{I_a}\\
    \end{split}
\end{equation}

Da mit den Messwerten die Ströme über den Spannungen abgebildet werden ist die
Steigung nicht der Widerstand sondern der Leitwert. Deshalb muss zur Ermittlung
des Ankwiderstand noch das Reziproke des Leitwerts berechnet werden.

\begin{equation} \label{eq2}
    \begin{split}
        G=\frac{1}{R}=\frac{I_a}{U_a}
    \end{split}
\end{equation}



\subsection{Ausgabe der Lösung}
\begin{figure}[H]
 \centering
 \includegraphics[width=1\textwidth]{as_labor01_2.png}
 \caption{Plot der Aufgabe 2}
 \label{fig:PlotAufgabe2}
\end{figure}

\subsection{Matlab Code}
\lstinputlisting[language=Matlab]{matlab/as_labor01_2.m}
