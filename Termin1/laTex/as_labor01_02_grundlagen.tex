\section{Grundlagen und Theorie}

\subsection{Methode der kleinsten Quadrate}
Die Methode der kleinsten Quadrate ist ein mathematisches Verfahren, bei dem eine lineare Regression
auf der Basis einer Wolke aus Datenpunkten berechnet werden soll. Es soll eine Kurve gefunden werden,
die möglichst nah an den Punkten verläuft. Dazu bestimmt man die Parameter dieser Kurve numerisch, indem die Summe 
der quadratischen Abweichungen der Kurve von den beobachteten Punkten minimiert wird.

Zur Umsetzung der Methode der kleinsten Quadrate in Matlab werden die Funktionen polyfit() und polyval() verwendet. 
polyfit() erhält beim Aufruf die Werte der Punktwolke sowie den Grad des Polynoms und gibt die entsprechenden Koeffizienten
zurück. polyval() ermittelt aus den Koeffizienten und den x-Werten die tatsächlichen Werte, mit welchen die Kurve geplottet 
werden kann. 

\subsection{Inkrementalgeber}
Ein Inkrementalgeber ist ein Messinstrument zur Ermittlung von Lage- oder Winkeländerung (bei rotierenden Objekten). 
Als verschiedene Arten wird zwischen der photoelektrischen Abtastung (entweder als abbildendes oder interferentielles
Messprinzip), der magnetischen Abtastung und per Schleifkontakt unterschieden. Dabei werden zwei um 90 Grad verschobene
Signale erzeugt, über die sich Drehgeschwindigkeit, -richtung und -winkel bestimmen lassen.
Im Beispielt der photoelektrischen Abtastung wird eine Drehscheibe verwendet, die mit mehreren Schlitzen unterteilt ist und 
zwischen einer Leuchtdiode und zwei leicht versetzten Photodetektoren angebracht ist. Wenn sich die Scheibe dreht, zählen
die Photodetektoren die Impulse, welche von Leuchtdiode und Lichtgitter der Drehscheibe entstehen.  