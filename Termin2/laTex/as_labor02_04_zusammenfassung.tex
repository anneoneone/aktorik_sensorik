\section{Zusammenfassung}

Mittels der Messwerte sind wir auf folgende Werte für die Ankerinduktivität $L$
und die Reibungskonstante $c_r$ gekommen.

\begin{equation} \label{eq241}
    \begin{split}
       L &=  175.462 \mathrm{\mu H}\\
       c_r &= 3,24 \cdot 10^{-9} \mathrm{\frac{Nm s}{rad}}
    \end{split}
\end{equation}

Die Berechnung der Reibungskonstante stellte uns vor keine Herausforderung
und ist sowohl Einheitenmäßig als auch Dimensionsmäßig nachvollziehbar.\\

Die Bestimmung der Ankerinduktivität stellte uns allerdings vor die
Herausforderung das System zu liniearisieren (Abbildung 2: Induktivität 1).
Nach der Liniearisierung der Funktion und anschließender Regression haben wir
den errechneten Wert der Induktivität wieder in die Ursprungsfunktion eingesetzt
(Abbildung 2: Induktivität 2). Hier ist zu erkennen, dass die ermittelte Gerade
sich nur bedingt in der Nähe der Messpunkte befindet. Wir vermuten, dass entweder
eine Ungenauigkeit in unseren Modell vorhanden ist oder es ist einen vielleicht
sogar frequenzabhängigen Offset in den Messdaten gibt. Um dieses Verhalten genauer
zu untersuchen, haben wir die Funktion und die Messdaten \href{https://www.desmos.com/calculator/2je7stqm76}{interaktiv in Desmos}
erstellt.\\

Wir konnten diese Diskrepanz leider nicht abschließend klären, sind aber zum Entschluss
gekommen das unser Ergebnis von $175.462 \mathrm{\mu H}$ ausreichend genau und auch realistisch ist.