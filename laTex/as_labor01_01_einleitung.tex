\section{Labor 1}

\subsection{Einleitung und Ziel}

Um später einen permanent erregten Gleichstrommotor zu modellieren, soll
in dieser ersten Laborübung in Aktor und Sensorik die wichtigsten
Kennwerte des Systems bestimmt werden. Dieses sind die Momentenkonstante
$k_M$, der Ankerwiderstand $R$, die Motorkonstante $k_e$ und der
Verstärkungsfaktor $A$ des Messverstärkers. 

Um diese Konstanten zu bestimmen wurden jeweils eine Menge an Messwerten
aufgenommen. Mit diesen wird mittels der Methode der kleinsten Quadrate 
Ausgleichsrechnungen durchgeführt. Dadurch erhalten wir eine lineare
Funktionen aus denen sich die gesuchten Konstanten bestimmen lassen.