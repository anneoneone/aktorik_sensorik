\subsection{Aufgabenstellung und Versuch}

Für das Subsystem Steuerung mit H-Brücke wurden folgenden Ein- und Ausgänge
definiert.\\

\begin{itemize}
    \item ($In_1$) Die Eingangs oder Betriebsspannung $U_H$
    \item ($In_2$) Der Momentanstrom $I_{ist}$
    \item ($In_3$) Der angestrebte Strom $I_{soll}$
    \item ($Out_1$) Die Spannung die über dem Motor abfällt $U_{Motor}$
\end{itemize}


\begin{figure}[H]
    \centering
    \includegraphics[width=1\textwidth]{hbridge_modell.png}
    \caption{Subsystem Steuerung mit H-Brücke}
    \label{fig:Subsystem H-Bridge}
\end{figure}

Das Subsystem beinhaltet Relay-Block der hier als Zweipunktregeler genutzt
wird. Die Beiden Schwellwerte $i_{plus}=10\mathrm{mA}$ und $i_{minus}=-10
\mathrm{mA}$ wurden im m-File definiert. Dieser gibt einen boolschen Wert
an den 1. Switch Block der zwischen Transistoren An und Aus hin- und
herschaltet. Der 2. Switch Block verändert die Drehrichtung des Motors und
wird geschaltet vom Vorzeichen des einzustellenden Strom $I_{soll}$.