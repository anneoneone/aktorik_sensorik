\section{Einleitung und Ziel}

Im 3. Labor des Moduls Aktorik und Sensorik soll ein mathematische Modell
eines Gleichstrommotors aufgestellt und in Simulink umgesetzt werden.\\

Die dafür nötigen Konstanten (u.a Ankerwiderstand, Induktivität, etc.)
sind in den beiden vergangenen Versuchen bereits bestimmt worden.
Um das Trägheits-moment $J$ -- die einzige fehlende Konstante im Modell --
zu bestimmen, wird das Modell mit einer Messung am realen System
verglichen. In dieser Messung wurde ein Spannungs-Sprung auf den Motor
gegeben und der Strom als Sprungantwort aufgenommen.


% \section{Einleitung und Ziel}

% Im 4. Labor im Modul Aktorik und Sensorik soll jubaefiubaefiu bestimmt werden.


% Was m"ussen Sie um das bestehende Motormodell hinzufügen, um einen Strom einprägen zu können?
% H-Br"ucke

% 2.1
% T_LH und T_RL

% 2.2 
% Durch die Drehung des Motors wird eine Spannung Induziert. Dadurch fließt ein Strom durch die 
% sogenannten Parasit"ardioden T_RH und T_LL. 

% 2.3
% Der Strom l"asst sich "uber R_shunt messen, welcher passenderweise mit 1 OHM gew"ahlt ist. So ist
% die Spannung, die an R_shunt abf"allt gleich des Stromes.

% 2.4
% Es gibt vier verschiedene Zust"ande. Wenn der Motor im Uhrzeigersinn betrieben werden soll, k"onnen
% die entsprechenden Transistoren geschaltet sein oder nicht geschaltet sein. Soll der Motor gegen
% den Uhrzeigersinn betrieben werden, kann das andere Transistorenpaar geschaltet sein oder nicht.

% 2.5
% Eing"ange: I_IST, I_Soll, U_H-Br"ucke
% Ausg"ange: U_Motor

% 2.6
% 1. Fall: AN
%     U_H = U_T_LH + U_Motor + U_T_RL + U_R_shunt

%     U_H = R_LH * I_m + (R_shunt * I_m)  

% 2. Fall: Aus
%     0 = U_H + U_D + U_Motor + U_D + U_shunt