\section{Labor 3}

\subsection{Einleitung und Ziel}

Im 3. Labor des Moduls Aktorik und Sensorik soll ein mathematische Modell
eines Gleichstrommotors aufgestellt und in Simulink umgesetzt werden.\\

Die dafür nötigen Konstanten (u.a Ankerwiderstand, Induktivität, etc.)
sind in den beiden vergangenen Versuchen bereits bestimmt worden.
Um das Trägheits-moment $J$ -- die einzige fehlende Konstante im Modell --
zu bestimmen, wird das Modell mit einer Messung am realen System
verglichen. In dieser Messung wurde ein Spannungs-Sprung auf den Motor
gegeben und der Strom als Sprungantwort aufgenommen.