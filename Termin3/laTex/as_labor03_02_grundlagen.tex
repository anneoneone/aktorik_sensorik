\section{Grundlagen und Theorie}

Das Blocksschaltbild, welches in Simulink umzusetzten ist, resultiert aus
den beiden folgenden DGL.

\begin{equation} \label{eq211}
    \begin{split}
        \dot{i(t)}&=\frac{1}{L} \left[ u(t) - (R + R_s) \cdot i(t) - ke \cdot \omega(t) \right]\\
        \dot{\omega(t)}&=\frac{1}{J} \left[km \cdot i(t) -C_r \omega(t) \right]
    \end{split}
\end{equation}

Die erste Gleichung ergibt sich aus der Maschengleichung des elektrischen Teils.
Der zweite Teil der DGL ist auf die Summation der Drehmomente zurückzuführen.